\subsubsection{UC-DPPS-130-1.2}

This UC describes the general features of the \texttt{ctapipe-process}
command-line tool that implements it and all of its child use cases. The
features described in the use case are implemented as part of the design of the
\texttt{ctapipe.core.Tool} framework, upon which this tool is based. Therefore
by inspection, and unit tests within the \texttt{ctapipe} package itself, we can
verify:

\begin{itemize}
  \item The tool uses standard return codes, and reports status in logs as well as in structured provenance information
  \item It generates local provenance log info in a structured format, including the exit status
  \item The exit status includes the case where the user or system stopped the job with a signal
  \item Logs are written to stdout or to files, with user-specified levels
  \item Configuration is handled by config files and command-line options for
        all exposed parameters.
  \item It defines an output data format that includes the necessary output data and metadata
  \item It is able to read simulation data, and has a plugin system  with packages to read other required inputs, currently including the ACADA ZFITS format.
  \item The user has an option to choose the target data level
  \item The tool contains a standardized analysis workflow that can be configured in detail that loops over events in the input and transforms them to the desired output.
  \item The user can get a sample configuration file (via a second tool), or get detailed help information on all potential parameters via the \texttt{--help} or \texttt{--help-all} options.
\end{itemize}

Therefore, along with the automatic tests for this UC, we can consider it verified.

\subsubsection{UC-DPPS-130-1.3}

The \emph{Train a Reconstrucion Model} use case describes how reconstruction models are trained in general, however the specific implementations are specified in UC-DPPS-130-2.1.1 (energy), UC-DPPS-130-2.1.2 (gammaness), and  UC-DPPS-130-2.1.3 (monoscopic displacement). Therefore the unit tests and workflows will be tested only when those UCs are included.  However, as the implementation already exists for these tools in \texttt{ctapipe}, by inspecting code e.g. for the \texttt{ctapipe-train-energy-regressor} tool, we can verify that the general UC is covered.  The \texttt{ctapipe} training framework provides:

\begin{itemize}
  \item input of gamma, electron, and proton events as training data
  \item configuration of the input features from a generic list
  \item the ability to specify event preselection parameters
  \item the ability to set a test/train split fraction
  \item the ability to generate cross-validation metrics
  \item the output of models that are readable by UC-DPPS-130-1.4 (Apply Reconstructino Models).
\end{itemize}

The unit tests of ctapipe already test the functionality of :
\begin{itemize}
  \item \texttt{ctapipe-train-energy-regressor}
  \item \texttt{ctapipe-train-particle-classifier}
  \item \texttt{ctapipe-train-disp-reconstructor}
\end{itemize}

Therefore we consider this UC verified, and specific tests are deferred until
the extended use-cases listed above are implemented as workflows.
