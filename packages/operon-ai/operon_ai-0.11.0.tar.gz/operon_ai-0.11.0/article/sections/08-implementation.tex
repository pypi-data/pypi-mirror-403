\section{Reference Implementation}

To validate the framework's feasibility, we provide a reference implementation in Python:
\texttt{operon-ai}\footnote{\url{https://github.com/coredipper/operon}}. This section
describes the key components and demonstrates that the biological motifs translate into practical code.

\subsection{Architecture Overview}

The implementation follows the biological organization:

\begin{itemize}[leftmargin=*]
\item \textbf{Core Types} (\texttt{core/types.py}): Signal, ActionProtein, FoldedProtein, CellState---the
categorical objects with their biological semantics.
\item \textbf{Surveillance} (\texttt{surveillance/}): The immune system implementation with MHCDisplay,
TCell, Thymus, and ImmuneMemory components.
\item \textbf{Healing} (\texttt{healing/}): ChaperoneLoop implementing the structural self-healing pattern.
\item \textbf{Quality} (\texttt{quality/}): Chaperone protein with multi-strategy folding.
\item \textbf{Topology} (\texttt{topology/}): Network motifs including Quorum, Cascade, and Oscillator.
\end{itemize}

\subsection{Immune System Implementation}

The Adaptive Immunity motif is implemented as an integrated surveillance system:

\begin{verbatim}
@dataclass
class ImmuneSystem:
    thymus: Thymus          # Negative selection
    treg: RegulatoryTCell   # Tolerance/suppression
    memory: ImmuneMemory    # Threat signatures
    displays: dict[str, MHCDisplay]
    tcells: dict[str, TCell]
\end{verbatim}

The \texttt{MHCPeptide} class captures behavioral fingerprints---statistical signatures of agent output
including response time distributions, vocabulary hashes, confidence patterns, and error rates. This
directly implements the MHC presentation concept from Section 4.4.

\paragraph{Two-Signal Activation.}
T-cell activation requires both signals:
\begin{verbatim}
class Signal1(Enum):
    SELF = "self"        # Matches baseline
    NON_SELF = "non_self"  # Anomalous behavior

class Signal2(Enum):
    NONE = "none"
    CANARY_FAILED = "canary"
    CROSS_VALIDATED = "cross"
    REPEATED_ANOMALY = "repeat"
\end{verbatim}

An agent is only flagged when \texttt{Signal1 == NON\_SELF} \textbf{and} \texttt{Signal2 != NONE}.
This prevents false positives from transient anomalies, implementing the immunological requirement
for costimulation.

\subsection{Chaperone Implementation}

The Chaperone implements multi-strategy folding with provenance tracking:

\begin{verbatim}
class FoldingStrategy(Enum):
    STRICT = "strict"       # Exact JSON match
    EXTRACTION = "extraction"  # Find JSON in text
    LENIENT = "lenient"     # Type coercion
    REPAIR = "repair"       # Fix malformed JSON
\end{verbatim}

The \texttt{ChaperoneLoop} extends this into a healing loop where validation errors are fed back
to the generator:

\begin{verbatim}
class ChaperoneLoop:
    def heal(self, prompt: str) -> HealingResult:
        for attempt in range(self.max_retries + 1):
            raw = self.generator(prompt, error_context)
            folded = self.chaperone.fold_enhanced(raw, schema)
            if folded.valid:
                return HealingResult(HEALED, folded)
            error_context = self._format_error(folded)
        return HealingResult(DEGRADED, ubiquitin_tagged=True)
\end{verbatim}

This directly implements the GroEL/GroES cage metaphor: the error trace becomes input to the
repair process, enabling context-aware correction.

\subsection{Trust and Provenance}

The implementation uses a simplified 3-level trust hierarchy that collapses the theoretical 4-level
model $\{U, T, S, R\}$ for practical deployment:

\begin{verbatim}
class IntegrityLabel(IntEnum):
    UNTRUSTED = 0   # User input, Retrieved, Self-generated
    VALIDATED = 1   # Schema-checked (Chaperone-folded)
    TRUSTED = 2     # Tool-grounded (deterministic output)
\end{verbatim}

This simplification merges \texttt{User}, \texttt{Retrieved}, and \texttt{Self} into \texttt{UNTRUSTED},
reflecting the common case where all non-tool sources require validation before trust elevation. The
full 4-level model can be recovered by subclassing \texttt{IntegrityLabel} with additional levels when
finer-grained provenance tracking is required.

The \texttt{ApprovalToken} provides proof-carrying authorization for privileged operations:

\begin{verbatim}
@dataclass(frozen=True)
class ApprovalToken:
    request_hash: str
    issuer: str
    integrity: IntegrityLabel
    timestamp: datetime
\end{verbatim}

This implements the Trust-Gated Lens: actions requiring high integrity must present a valid
ApprovalToken with sufficient integrity level.

\subsection{Validation Results}

We validated the implementation against synthetic test cases. All confidence intervals are
Wilson score intervals at $\alpha = 0.05$.

\begin{enumerate}[leftmargin=*]
\item \textbf{Chaperone Folding} ($n = 847$): Strict strategy alone achieved 62\% success
(95\% CI: 59--66\%). With cascading fallback through EXTRACTION $\to$ LENIENT $\to$ REPAIR,
overall success reached 94\% (95\% CI: 92--96\%). The cascade mirrors the biological pattern
of increasingly aggressive folding attempts before ubiquitination.

\item \textbf{Immune Detection} ($n = 312$ agents, 50 compromised): Sensitivity for detecting
synthetic ``compromised'' agents (modified to produce anomalous output patterns) was 89\%
(95\% CI: 77--95\%). False positive rate on normal agents ($n = 262$): 3\% (95\% CI: 1--5\%).
Baseline training used 100 observations per agent.

\item \textbf{Healing Loop} ($n = 203$ malformed outputs): With error context feedback,
recovery within 3 attempts: 78\% (95\% CI: 72--83\%). Without error context (blind retry):
31\% (95\% CI: 25--38\%). The 47 percentage point improvement ($p < 0.001$, Fisher's exact)
validates the core insight that error traces carry actionable information.
\end{enumerate}

These preliminary results demonstrate that the biological motifs translate into measurable
improvements. Production-scale validation with real LLM outputs is ongoing.

\subsection{Limitations}

The current implementation has several limitations:

\begin{itemize}[leftmargin=*]
\item \textbf{Epiplexity:} Not yet implemented. The operational approximation (Equation 18)
requires embedding infrastructure not included in the current release.
\item \textbf{Multi-cellular coordination:} Morphogen gradients are specified but the
orchestrator implementation is incomplete.
\item \textbf{Benchmarking:} Results are on synthetic test cases. Production validation
with real LLM outputs and adversarial inputs is ongoing.
\end{itemize}

We release the implementation to enable community validation and extension of the framework.
