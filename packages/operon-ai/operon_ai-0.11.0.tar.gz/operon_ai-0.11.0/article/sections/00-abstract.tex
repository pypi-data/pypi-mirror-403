\begin{abstract}
The transition of Large Language Models (LLMs) from passive generators to autonomous agents
has introduced significant challenges in reliability, security, and state management. Current agentic
architectures are often constructed ad-hoc, prone to ``hallucination cascades,'' infinite loops, and
prompt injection attacks. This paper proposes that these failure modes are not unique to software
but are instances of universal control problems solved by biological systems over billions of years.

We present a formal isomorphism, \emph{at the level of their polynomial-interface models}, between Gene
Regulatory Networks (GRNs) and Agentic Software Systems using Applied Category Theory. We model
agents as Polynomial Functors within the category $\mathbf{Poly}$, and their interactions via the Operad
of Wiring Diagrams. We derive a rigorous syntax for agent composition by mapping biological
mechanisms---including Quorum Sensing for consensus, Chaperone Proteins for structural validation,
Innate and Adaptive Immunity for layered security, Mitochondrial Signaling for bioenergetic resource
governance, and Endosymbiosis for neuro-symbolic integration---to software design patterns. This framework
provides a mathematical basis for ``Epigenetic'' state management (RAG), a \textit{Provenance Functor} for
injection resistance, \textit{Epiplexity} for detecting epistemic stagnation, and a \textit{Metabolic
Coalgebra} that ensures decidable termination. A reference implementation validates the framework's
practical feasibility.
\end{abstract}
