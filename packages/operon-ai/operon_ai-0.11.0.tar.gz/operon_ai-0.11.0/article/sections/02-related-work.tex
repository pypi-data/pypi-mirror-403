\section{Related Work}

This work sits at the intersection of Systems Biology, Applied Category Theory, and Agentic AI. While significant
research exists within each domain, the formal synthesis of biological control topologies with agentic software
architectures has received limited attention.

\subsection{Network Motifs in Systems Biology}

The concept of ``Network Motifs''---statistically over-represented sub-graphs in complex networks---was introduced
by Milo et al.~\cite{milo2002network}. Their work demonstrated that biological networks are not random but are composed of specific
building blocks selected for functional data processing. Alon~\cite{alon2007network} further characterized the dynamical properties
of these motifs, identifying the Coherent Feed-Forward Loop (CFFL) as a persistence detector. We extend this by
mapping these motifs to the stochastic nature of Generative AI.

\subsection{Applied Category Theory (ACT)}

To formalize network structure, we draw upon ACT. Spivak~\cite{spivak2021learners} and Vagner et al.~\cite{vagner2015algebras} established a rigorous framework
for modeling Open Dynamical Systems using the category $\mathbf{Poly}$ and the Operad of Wiring Diagrams. To our
knowledge, this is the first application of Polynomial Functors specifically designed to model the interface of
LLM Agents and to verify safety properties in Agentic topologies.

\subsection{Reliability in Agentic AI}

Techniques such as ``Chain of Thought''~\cite{wei2022chain} utilize iterative looping to improve output quality. However, these
methods operate primarily at the level of the prompt (the input signal) rather than the topology (the wiring).
By importing the concept of Autopoiesis~\cite{maturana1980autopoiesis}, we propose a methodology where reliability is a property of the
network architecture itself.
