\section{The Mapping: Biology $\leftrightarrow$ Software}

To treat Agentic Systems and Gene Regulatory Networks (GRNs) as isomorphic \textbf{at the level of their typed
interfaces}, we must map them to a common mathematical object. We utilize the category $\mathbf{Poly}$, where
objects are polynomial functors representing interfaces, and morphisms represent interaction protocols.

\subsection{Preliminaries: The Category $\mathbf{Poly}$}

In Applied Category Theory, a Polynomial Functor $P$ represents a typed interface for a dynamical system. It is
defined as a sum of representable functors:
\begin{equation}
P(y) = \sum_{o\in O} y^{I_o}.
\tag{2}
\end{equation}
Here, $O$ is the set of possible Positions (or Outputs) the system can expose. For each position $o\in O$, there
is a set $I_o$ of Directions (or Inputs) required to transition the system to a new state.
\begin{itemize}[leftmargin=*]
\item The coefficient $o$ represents the value produced by the system.
\item The exponent $I_o$ represents the capacity to receive information from the environment.
\end{itemize}
This formalism captures the essence of a ``stateful interface'': the system outputs a value $o$ and then waits
for a specific type of input $i\in I_o$ before it can proceed.

\begin{figure}[h]
\centering
\begin{tikzpicture}[>=Latex, node distance=12mm]
  \node[draw, rounded corners, minimum width=30mm, minimum height=10mm] (cap) {Output $o\in O$};
  \node[draw, circle, below left=10mm and 10mm of cap] (i1) {$i_1\in I_o$};
  \node[draw, circle, below=10mm of cap] (i2) {$i_2\in I_o$};
  \node[draw, circle, below right=10mm and 10mm of cap] (i3) {$\cdots$};
  \draw[->] (i1) -- (cap);
  \draw[->] (i2) -- (cap);
  \draw[->] (i3) -- (cap);
  \node[below=12mm of i2] {The Interface $P(y)$};
\end{tikzpicture}
\caption{A visual representation of a Polynomial Functor (often called a ``Mushroom'' or ``Corolla''). The system
offers an Output (the cap) and exposes specific Input ports (the stalks) dependent on that output.}
\end{figure}

\subsection{The Isomorphism: Genes and Agents}

We now apply this abstract definition to our specific domains.

\paragraph{Definition 1 (The Gene Object).}
A gene $G$ is a polynomial functor where $O_G$ is the set of expressed proteins and
$I_G=(I_{\text{prot}})_{\text{prot}\in O_G}$ is the \textbf{family} of regulatory-signal sets (transcription factors)
available at each expressed protein:
\begin{equation}
P_{\text{Gene}}(y) = \sum_{\text{prot}\in \text{Proteins}} y^{I_{\text{prot}}}.
\tag{3}
\end{equation}

\paragraph{Definition 2 (The Agent Object).}
An autonomous agent $A$ is a polynomial functor where $O_A$ is the set of generated messages/actions, and
$I_A=(I_{\text{action}})_{\text{action}\in O_A}$ is the \textbf{family} of observation sets available at each action:
\begin{equation}
P_{\text{Agent}}(y) = \sum_{\text{action}\in \text{Actions}} y^{I_{\text{action}}}.
\tag{4}
\end{equation}

\subsection{The Interface: Promoters as Lenses}

In biology, a gene is not universally accessible. It is guarded by a Promoter Region---a specific DNA sequence
that only binds to compatible Transcription Factors. In software, an agent is guarded by an API Schema or
Context Window definition.

We model this gating mechanism using Optics, specifically Lenses. A Lens consists of two maps between a global
state $S$ and a local view $V$:
\begin{enumerate}[leftmargin=*]
\item Get (View): $\mathrm{get}: S \to V$ (Extracting relevant signal from global state).
\item Put (Update): $\mathrm{put}: S\times V' \to S$ (Updating global state based on local change).
\end{enumerate}

The ``Promoter'' acts as a filter that determines which part of the global cellular environment ($S$) is visible
($V$) to the gene.
\begin{itemize}[leftmargin=*]
\item \textbf{Biological Lens:} The promoter filters the chaotic cellular soup, allowing the gene to ``see''
only specific molecules (e.g., Lac Repressor).
\item \textbf{Agentic Lens:} The Context Window filters the massive vector database, allowing the agent to
``see'' only the relevant retrieved chunks (RAG).
\end{itemize}

If the input signal does not match the Schema (Promoter), the Lens fails to focus, and the interaction is routed
to an explicit \textbf{inactive/error} case (equivalently, one works with a \textbf{partial} lens, or a total lens
into $V+\mathrm{Error}$) (the agent does not run; the gene is not expressed).

\subsection{Epigenetics and State: The Coalgebra}

Neither genes nor agents are stateless functions. They possess memory.
\begin{itemize}[leftmargin=*]
\item \textbf{Biology:} Epigenetic markers (Methylation, Histone modification) alter how a gene responds to
signals without changing the DNA code.
\item \textbf{Software:} Retrieval Augmented Generation (RAG) and Conversation History alter how an agent
responds to a prompt without changing the LLM weights.
\end{itemize}

We model this as a Coalgebra for the polynomial functor $P$. A dynamical system is defined as a tuple $(S,\phi)$,
where $S$ is the state space and $\phi$ is the structure map:
\begin{equation}
\phi: S \to P(S).
\tag{5}
\end{equation}

By expanding $P(S)$, we derive the two fundamental operations of the state machine:
\begin{enumerate}[leftmargin=*]
\item Readout: $S \to O$ (Given current state/memory, what action do I take?)
\item Update: $\displaystyle \sum_{s\in S} I_{o(s)} \to S$ (Given current state $s$ and a new input
$i\in I_{o(s)}$ compatible with its current output $o(s)$, what is my new state?)
\end{enumerate}

By establishing this formal dictionary (Table 1), we \textbf{can regard} GRNs and Agentic Systems as distinct
implementations of the same abstract dynamical system \textbf{under this interface-level abstraction}.

\begin{figure}[h]
\centering
\begin{tabular}{@{}c@{\qquad}c@{}}
\begin{tabular}{@{}c@{}}
Gene\\(Function)\\[0.25em]
Transcription Factors ($I$)\\
Proteins ($O$)\\
Promoter Binding\\
Expression
\end{tabular}
&
\begin{tabular}{@{}c@{}}
Agent\\(LLM + Tools)\\[0.25em]
Observations ($I$)\\
Actions ($O$)\\
Schema Match\\
Generation
\end{tabular}
\end{tabular}
\\[0.5em]
\caption{The Structural Isomorphism. Both Genes and Agents act as transducers converting Input Contexts ($I$) into
Output Expressions ($O$), governed by the same categorical laws (at the level of polynomial-interface models).}
\end{figure}

\begin{table}[h]
\centering
\begin{tabular}{@{}lll@{}}
\toprule
Category Concept & Biological Realization (GRN) & Software Realization (Agentic)\\
\midrule
Polynomial Functor ($P$) & Gene Interface & Agent Interface (System Prompt)\\
Output Position ($O$) & Protein Expression & Tool Call / Message\\
Input Direction ($I$) & Transcription Factor Binding & Observation / User Prompt\\
Lens (Optic) & Promoter Region & API Schema / Context Window\\
Internal State ($S$) & Epigenetic Markers (Methylation) & Vector Store / Chat History\\
Morphism ($\circ$) & Signal Transduction Pathway & Data Pipeline\\
\midrule
\multicolumn{3}{@{}l@{}}{\textit{Organelles (Specialized Processing Units)}}\\
\midrule
Template Engine & Ribosome (mRNA $\to$ Protein) & Prompt Template Factory\\
Output Validation & Chaperone (Protein Folding) & Schema Validator / JSON Parser\\
Waste Processing & Lysosome (Autophagy) & Error Handler / Garbage Collector\\
Decision Center & Nucleus (Transcription) & LLM Provider Wrapper\\
Input Filter & Membrane (Immune System) & Prompt Injection Defense\\
Computation Engine & Mitochondria (MIPS) & Runtime Supervisor / Decision Gate\\
\midrule
\multicolumn{3}{@{}l@{}}{\textit{Lifecycle and Rhythms}}\\
\midrule
Lifespan Limit & Telomere Shortening & Operation Counter / Max Iterations\\
Periodic Scheduling & Circadian Oscillator & Health Checks / Heartbeat\\
\bottomrule
\end{tabular}
\caption{The Isomorphism Dictionary (Extended)}
\end{table}

\begin{table}[h]
\centering
\begin{tabular}{@{}lll@{}}
\toprule
\textbf{Biological Concept} & \textbf{Agentic Concept} & \textbf{Formal Structure}\\
\midrule
\multicolumn{3}{@{}l@{}}{\textit{Immunity and Security}}\\
\midrule
Toll-Like Receptor (TLR) & Regex Injection Filter & Pattern Matching\\
MHC Presentation & Provenance Labeling & Functor $\mathcal{P}: \mathbf{Msg} \to \mathbf{Trust}$\\
T-Cell Receptor & Trust Gate & Partial Lens\\
Negative Selection & Injection Training & Penalized Learning\\
Regulatory T-Cell & Confidence Dampening & Suppression Function\\
Immune Memory & Threat Signature Store & Hash-Indexed Cache\\
\midrule
\multicolumn{3}{@{}l@{}}{\textit{Metabolism and Control}}\\
\midrule
ATP & Token Budget & Resource Monoid $\mathcal{R}$\\
mPTP Opening & Fast Apoptosis Trigger & Guard Condition\\
Retrograde Signaling & Phenotype Reshaping & Slow Adaptation\\
Metabolic-Epigenetic Coupling & Cost-Gated Retrieval & Conditional Access\\
\midrule
\multicolumn{3}{@{}l@{}}{\textit{Information and Health}}\\
\midrule
Trophic Factors & Novel Input Signals & Epiplexity $> \delta$\\
Bayesian Brain & Free Energy Minimization & KL Divergence\\
Apoptosis & Agent Termination & State $\to \bot$\\
\midrule
\multicolumn{3}{@{}l@{}}{\textit{Multi-Cellular Organization}}\\
\midrule
Genome & Base Model Weights & Shared Parameters\\
Epigenome & System Prompt + RAG & Phenotype Context\\
Morphogen Gradient & Shared Context Variables & JSON State\\
Tissue Boundary & Trust Boundary & Type Barrier\\
\bottomrule
\end{tabular}
\caption{Extended Isomorphism Dictionary: Security, Metabolism, and Organization}
\end{table}

\subsection{Metabolic Coalgebras: Formalizing Resource Constraints}

Finally, we address the physical constraints of computation. Just as biological systems are limited by ATP availability \cite{lynch2015bioenergetic}, agentic systems are limited by token budgets and latency. To model this, we extend our coalgebraic framework to include resource constraints, defining a \textbf{Metabolic Coalgebra}. Mathematically, this is an instance of a Quantitative Coalgebra enriched over a resource monoid, effectively restricting the domain of the state transition function to resource-sufficient states.

We align this definition with the theory of \textbf{Quantitative Polynomial Functors} \cite{nakov2021quantitative}, treating the system as a state machine enriched over a resource monoid.

\begin{definition}[The Resource Monoid]
Let $(\mathcal{R}, +, 0, \ge)$ be an ordered commutative monoid representing computational resources (e.g., token counts), where $\mathcal{R} \cong \mathbb{N}$.
\end{definition}

\begin{definition}[Metabolic Coalgebra]
A resource-constrained agent is a coalgebra $(S, \alpha)$ over a polynomial functor $P$, where the state space is the product of the logical state $L$ and the resource state $\mathcal{R}$:
\begin{equation}
    S \cong L \times \mathcal{R}
\end{equation}
The structure map $\alpha: S \to P(S) + \bot$ is defined as a \textbf{partial map} guarded by cost. For a transition requiring cost $c \in \mathcal{R}$:
\begin{equation}
    \alpha(l, r) = 
    \begin{cases} 
      (l', r - c) & \text{if } r \ge c \\
      \bot & \text{if } r < c \quad \text{(Apoptosis)}
    \end{cases}
\end{equation}
\end{definition}

This structure maps to the energetics of transcriptional elongation. A gene (Agent) cannot express its protein (Action) instantaneously; it must transcribe an mRNA sequence (Chain of Thought) nucleotide by nucleotide. This process consumes a distinct amount of chemical energy (NTPs) per step. The Metabolic Coalgebra models this dependency: if the cellular energy budget is exhausted, transcription stalls (Ischemia), and the gene fails to execute its function, regardless of its regulatory logic.

This formalism establishes that ``Ischemia'' (Token Starvation) is not merely a runtime error, but a reachable terminal state $\bot$ in the system's dynamics. This mirrors the biological mechanism where failure to meet metabolic costs triggers p53-mediated apoptosis \cite{aubrey2018p53}.


\begin{theorem}[The Metabolic Bound]
For any agentic topology $T$ composed of $N$ agents with total budget $R_{total}$, the system is guaranteed to halt. Unlike the general Halting Problem, termination is decidable for Metabolic Coalgebras: the resource state $r$ is strictly decreasing for every non-identity morphism, providing a well-founded termination measure \cite{boreale2023weighted}.
\end{theorem}

\subsection{Additional Organelles: Completing the Cellular Architecture}

Beyond the core gene-agent mapping, biological cells contain specialized organelles that handle distinct aspects of cellular function. We extend our isomorphism to four additional structures that map directly to agentic software components.

\paragraph{Ribosome: Template-to-Output Synthesis.}
In biology, the ribosome reads messenger RNA (mRNA) sequences and synthesizes proteins by assembling amino acids according to the genetic code. Transfer RNA (tRNA) molecules carry amino acids to the ribosome, where codons (three-nucleotide sequences) specify which amino acid to add.

In agentic systems, the Ribosome maps to a \textbf{prompt template engine}:
\begin{itemize}[leftmargin=*]
\item \textbf{mRNA} $\to$ Prompt templates with variable slots
\item \textbf{tRNA} $\to$ Context bindings (variable $\to$ value mappings)
\item \textbf{Codons} $\to$ Template directives (variables, conditionals, loops)
\item \textbf{Translation} $\to$ Template rendering with context injection
\end{itemize}
Just as the ribosome ensures that the genetic code is faithfully translated into functional proteins, the software ribosome ensures that abstract prompt templates are instantiated into concrete, well-formed prompts.

\paragraph{Lysosome: Waste Processing and Recycling.}
The lysosome is the cell's recycling center, containing enzymes that break down cellular waste, damaged organelles, and foreign material. Through autophagy, the cell digests its own components to recover building blocks during stress.

In agentic systems, the Lysosome maps to \textbf{error handling and garbage collection}:
\begin{itemize}[leftmargin=*]
\item \textbf{Waste Classification} $\to$ Categorizing failures (timeout, validation error, toxic input)
\item \textbf{Digestion} $\to$ Processing errors to extract debugging information
\item \textbf{Recycling} $\to$ Recovering useful context from failed operations
\item \textbf{Autophagy} $\to$ Periodic cleanup of stale cache and expired state
\item \textbf{Toxic Disposal} $\to$ Secure handling of sensitive data (API keys, PII)
\end{itemize}
The lysosome prevents accumulation of ``cellular debris'' that could poison the system---analogous to memory leaks or error cascades in software.

\paragraph{Nucleus: The Decision Center.}
In eukaryotic cells, the nucleus houses the DNA and serves as the control center for gene expression. Transcription factors enter the nucleus, bind to promoter regions, and initiate transcription of specific genes.

In agentic systems, the Nucleus maps to the \textbf{LLM provider wrapper}:
\begin{itemize}[leftmargin=*]
\item \textbf{DNA} $\to$ Pre-trained model weights (static substrate of capability)
\item \textbf{Transcription} $\to$ Inference (prompt $\to$ response generation)
\item \textbf{Nuclear Envelope} $\to$ Provider abstraction layer (API boundary)
\item \textbf{Nucleolus} $\to$ Tool integration hub (where external capabilities are assembled)
\end{itemize}
The nucleus abstracts the complexity of the underlying LLM, exposing a consistent interface regardless of the provider (Anthropic, OpenAI, Gemini).

\paragraph{Telomere: Lifecycle and Senescence.}
Telomeres are protective caps at the ends of chromosomes that shorten with each cell division. When telomeres become critically short, the cell enters senescence (permanent growth arrest) or apoptosis. The enzyme telomerase can extend telomeres, enabling continued division in stem cells.

In agentic systems, Telomeres map to \textbf{lifecycle management}:
\begin{itemize}[leftmargin=*]
\item \textbf{Telomere Length} $\to$ Remaining operation budget (max iterations)
\item \textbf{Shortening} $\to$ Decrementing counter per operation
\item \textbf{Senescence} $\to$ Graceful degradation (reduced capability mode)
\item \textbf{Apoptosis} $\to$ Clean shutdown when budget exhausted
\item \textbf{Telomerase} $\to$ Renewal mechanism (resetting counters for trusted agents)
\end{itemize}
This provides a biological basis for the common pattern of limiting agent iterations. Rather than arbitrary timeouts, the telomere model frames lifecycle limits as a natural property of the system's ``cellular age.''

\paragraph{Mitochondria: The Metabolic Motherboard.}
Recent scholarship reframes mitochondria not merely as the cell's powerhouse, but as the \textbf{Mitochondrial Information Processing System (MIPS)} \cite{picard2022mips}. Mitochondria sense environmental stress and integrate metabolic signals to govern cellular decision-making. They function as ``social signaling organelles'' that communicate with the nucleus and other organelles.

In agentic systems, the Mitochondria maps to the \textbf{Runtime Supervisor}:
\begin{itemize}[leftmargin=*]
\item \textbf{ATP Production} $\to$ Deterministic computation (tool execution, code evaluation)
\item \textbf{Stress Sensing} $\to$ Monitoring token burn rate vs. informational yield
\item \textbf{Retrograde Signaling} $\to$ Forcing strategy shifts when metabolic efficiency drops
\item \textbf{Fusion/Fission} $\to$ Context fusion under resource constraints
\end{itemize}

\textbf{Temporal Dynamics: Fast vs. Slow Interventions.}
A crucial distinction exists between the timescales of mitochondrial intervention. In biology, retrograde signaling influences nuclear gene expression over hours to days---it is a \textit{developmental} response that reshapes the cell's phenotype. In contrast, acute metabolic stress (ATP depletion, Ca$^{2+}$ overload) triggers immediate responses: cytochrome c release initiates apoptosis within minutes.

We preserve this distinction in the agentic mapping:
\begin{itemize}[leftmargin=*]
\item \textbf{Fast Intervention (Acute):} The Runtime monitors real-time metrics (token velocity, error rate). When thresholds are breached, it triggers immediate Apoptosis---terminating the current chain-of-thought without negotiation. This mirrors the mitochondrial permeability transition pore (mPTP) opening.
\item \textbf{Slow Intervention (Chronic):} The Runtime accumulates statistics across sessions (average efficiency, failure patterns). These inform \textit{Retrograde Responses}---modifications to system prompts, retrieval strategies, or model selection that reshape the agent's ``phenotype'' over deployment cycles. This mirrors how chronic metabolic stress induces mitochondrial biogenesis and metabolic reprogramming.
\end{itemize}
The Runtime does not ``override'' the LLM in the sense of injecting tokens mid-generation; rather, it governs the \textit{boundary conditions} within which generation occurs, and triggers state transitions (continue, pivot, terminate) at defined checkpoints.
